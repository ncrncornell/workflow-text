% Flowcharting techniques for easy maintenance
% Author: Brent Longborough
% http://www.texample.net/media/tikz/examples/TEX/flexible-flow-chart.tex
\documentclass[x11names]{article}
\usepackage[usenames,dvipsnames]{xcolor}
\usepackage{tikz}
\usepackage{courier}
\usepackage{listings}
\lstset{basicstyle=\ttfamily}
\usepackage{hyperref}
\hypersetup{colorlinks=true,urlcolor=black,filecolor=black,linkcolor=black,citecolor=black,linkcolor=black}
\usepackage{attachfile}
\attachfilesetup{color=0 0 0}
\usetikzlibrary{shapes,arrows,chains,positioning,fit,calc,backgrounds}

\usepackage[printonlyused]{acronym}
\usepackage{fullpage}

\usepackage[backend=biber]{biblatex}
\bibliography{paper} % or

%%%<
\usepackage{verbatim}
%\usepackage[active,tightpage]{preview}
%\PreviewEnvironment{tikzpicture}
%\setlength\PreviewBorder{5mm}%
\newcommand{\repec}{RePEc}
%%%>
\newcommand{\mylstinputlisting}[2]{
\lstinputlisting[]{#1}
\textattachfile{#1}{(#2 : #1)}
}

% $Id: lstprov.tex 1272 2014-12-19 22:55:06Z lv39 $
\lstset{emph={%  
    prov:entity, prov:agent, prov:id, prov:ref, prov:document, prov:activity, 
    prov:wasAssociatedWith, prov:used, prov:wasGeneratedBy%
    },emphstyle={\color{Blue}\bfseries}%
}%

%\lstset{emph={%  
 %   dc:title, dc:date%
  %  },emphstyle={\color{CadetBlue}\bfseries}%
%}%

\title{An ex-post workflow documentation tool}
\author{Lars Vilhuber%
\footnote{Cornell University, corresponding author. This work is funded by NSF Grant \href{http://www.nsf.gov/awardsearch/showAward.do?AwardNumber=1131848}{1131848}.}
\and Carl Lagoze%
\footnote{University of Michigan}
\and Ben Perry%
\footnote{Cornell University}
}

\begin{document}
\maketitle
% =================================================
% Set up a few colours
\colorlet{lcadvisor}{Green3}
\colorlet{lcauthor}{Blue3}
\colorlet{lcuse}{Red3}
\colorlet{lcattr}{Green3}
\definecolor{ncentity}{HTML}{B0C4DE}
% -------------------------------------------------
% Set up a new layer for the debugging marks, and make sure it is on
% top
\pgfdeclarelayer{marx}
\pgfsetlayers{main,marx}
% A macro for marking coordinates (specific to the coordinate naming
% scheme used here). Swap the following 2 definitions to deactivate
% marks.
%\providecommand{\cmark}[2][]{%
%  \begin{pgfonlayer}{marx}
%    \node [nmark] at (c#2#1) {#2};
%  \end{pgfonlayer}{marx}
%  } 
\providecommand{\cmark}[2][]{\relax} 
% -------------------------------------------------
% Start the picture
%\include{tikz-example}
\section{Introduction}
\label{sec:intro}
% $Id: introduction.tex 1272 2014-12-19 22:55:06Z lv39 $
Describe PROV.\url{http://www.w3.org/TR/prov-primer/} Describe CED$^2$AR. Describe problem.

The basic empirical scientific workflow is straightforward. Collect or acquire data, establish an analysis protocol and execute via statistical software, obtain results, and discuss results in the form of technical documents or published journal articles. 

With this tool and associated implementation of a provenance-tracking document, we aim to capture the most frequent scenarios of that workflow. We start with the premise of an interested or co-erced user. For instance, take a researcher who is preparing the final manuscript for a journal that requires replication-ready archives of data and programs. If he hasn't done so up to this point, the researcher will start by collecting all the relevant pieces of data and code that make up the inputs to the paper. At its simplest, the paper has one table, that is produced by a single program (we assume that it's not produced by error-prone manual manipulations of data). That program read in some original data. See Figure~\ref{fig:workflow:simplistic}.
\begin{figure}[!h]
\centering
\caption{Provenance}\label{fig:workflow:simplistic}
% Flowcharting techniques for easy maintenance
% Author: Brent Longborough
% http://www.texample.net/media/tikz/examples/TEX/flexible-flow-chart.tex
% Start the picture
\begin{tikzpicture}[%
    >=triangle 60,              % Nice arrows; your taste may be different
    start chain=going below,    % General flow is top-to-bottom
    node distance=10mm and 60mm, % Global setup of box spacing
    every join/.style={author},   % Default linetype for connecting boxes
    ]
% ------------------------------------------------- 
% A few box styles 
% <on chain> *and* <on grid> reduce the need for manual relative
% positioning of nodes
\tikzset{
  base/.style={draw, on chain, on grid, align=center, minimum height=4ex},
  activity/.style={base, rectangle, text width=8em, minimum height=2em},
  agent/.style={base, diamond, aspect=2, text width=5em},
  entity/.style={base, ellipse,fill=ncentity!25,minimum height=2em,minimum width=8em,align=center},
  legend/.style={base, draw=none,rectangle},
  % coord node style is used for placing corners of connecting lines
  coord/.style={coordinate, on chain, on grid, node distance=6mm and 25mm},
  % nmark node style is used for coordinate debugging marks
  nmark/.style={draw, cyan, circle, font={\sffamily\bfseries}},
  % -------------------------------------------------
  % Connector line styles for different parts of the diagram
  author/.style={->, draw, lcauthor},
  adivsor/.style={->, draw, lcadvisor},
  use/.style={->, draw, lcuse},
  attr/.style={->, draw, lcattr},
  it/.style={font={\small\itshape}}
}
% -------------------------------------------------
% Legend
% Start by placing the nodes
\node [entity, densely dotted, it] (e0) {Paper};
% Use join to connect a node to the previous one 
\node [entity, densely dotted, it] (e1) {Table 1};
\node [entity, densely dotted, it] (e2) {program.do};
\node [entity, densely dotted, it] (e3) {CPS};
% We position the next block explicitly as the first block in the
% second column.  The chain 'comes along with us'. The distance
% between columns has already been defined, so we don't need to
% specify it.
\node [agent, , right=of e1] (p0) {Researcher};


% -------------------------------------------------
% Now we place the coordinate nodes for the connectors with angles, or
% with annotations. We also mark them for debugging.
%\node [coord, left=40mm of e0] (c0)  {}; \cmark{0}   
%\node [coord, left=40mm of p2] (c1)  {}; \cmark{1}   
%\node [coord, right=60mm of e0] (c2)  {}; \cmark{2}   
%\node [coord, right=20mm of c2] (c3)  {}; \cmark{3}   
%\node [coord, right=20mm of p5] (c4)  {}; \cmark{4}   
%\node [coord, right=10mm of c2] (c3)  {}; \cmark{3}   
% -------------------------------------------------
% A couple of boxes have annotations
%\node [above=0mm of p4, it] {(Queue was empty)};
%\node [above=0mm of p8, it] {(Queue was not empty)};
% -------------------------------------------------
% All the other connections come out of tests and need annotating
% First, the straight north-south connections. In each case, we first
% draw a path with a (consistently positioned) annotation node, then
% we draw the arrow itself.
\path (e0.south) to node [left] {contains} (e1);
  \draw [->,lcuse] (e0.south) -- (e1);
\path (e1.south) to node [left] {produced by} (e2);
  \draw [->,lcuse] (e1.south) -- (e2);
\path (e2.south) to node [left] {used} (e3);
  \draw [->,lcuse] (e2.south) -- (e3);
\end{tikzpicture}

\end{figure}

The goal of the software described in this paper is to ``interview'' the author about all the relevant components. Given the existing paper, what is the name, location, and purpose of the program that generated Table 1? What dataset served as input to the program? Where is that dataset located? And could you please describe in more detail the variables that make up your table? 

We choose a data-centric approach. Tying these concepts together, we leverage DDI for the data documentation (describe DDI, describe CED$^2$AR), and we leverage PROV (link to PROV) to describe the entities contributing to the program and the generated data. In its simplest instantiation, the input data is already well-catalogued, well-curated, and is browsable by DDI at a stable URL, possibly via a DOI. 

We build on our prior work incorporating PROV into DDI. We now implement that specification as a software tool that can be used to visualize the workflow, and also provide all the necessary information to (potentially) upload the data and programs to a publication site.

Issues that are solved including identifying the researcher (agent), specifying roles. 

The resulting DDI+PROV file is machine-readable, and complies with two widely used standards. An immediate application is the frequent task researchers encounter in restricted-access data environments. There, researchers need to document the data they are requesting release for in a way that the disclosure avoidance analysis (most often a human and not a program task) can understand. Typical documents require description of the data, of the programs, and where the data was sourced from - all elements of the process and documentation described here. (Census RDC, IAB, Synthetic Data Server).



\section{Basic PROV of a simple workflow}
\section{Graph components}
The completed file can be found as an attachment to this document 
(\textattachfile{prov-full.xml}{here}).
\lstset{language=XML,breaklines=true}
\subsection{Research activity}
\subsubsection*{Agents}
Agents undertake a research activity. Agents can be identified by external identifiers. In this 
example, we have identified users by their \repec~ handle, and included their \repec~ 
homepage. 

\lstinputlisting[firstline=65,lastline=76]{prov-full.xml}

\subsubsection*{Entities}
Entities are the datasets, programs, and articles that are being linked here.  A subset are listed 
here.
\lstinputlisting[firstline=15,lastline=23]{prov-full.xml}
\lstinputlisting[firstline=42,lastline=56]{prov-full.xml}
\subsubsection*{Activities}
We define a research activity to generate papers and data, and ultimately articles.
\lstinputlisting[firstline=78,lastline=83]{prov-full.xml}

\subsubsection*{Linking them}
\lstinputlisting[firstline=85,lastline=118]{prov-full.xml}

The straightforward research activity as traditionally focusing on papers would look like  
Figure~\ref{fig:author:complex}.
\begin{figure}
\caption{Authorship with research activity}\label{fig:author:complex}
% Flowcharting techniques for easy maintenance
% Author: Brent Longborough
% http://www.texample.net/media/tikz/examples/TEX/flexible-flow-chart.tex
% Start the picture
\begin{tikzpicture}[%
    >=triangle 60,              % Nice arrows; your taste may be different
    start chain=going below,    % General flow is top-to-bottom
    node distance=10mm and 60mm, % Global setup of box spacing
    every join/.style={author},   % Default linetype for connecting boxes
    ]
% ------------------------------------------------- 
% A few box styles 
% <on chain> *and* <on grid> reduce the need for manual relative
% positioning of nodes
\tikzset{
  base/.style={draw, on chain, on grid, align=center, minimum height=4ex},
  activity/.style={base, rectangle, text width=8em, minimum height=2em},
  agent/.style={base, diamond, aspect=2, text width=5em},
  entity/.style={base, ellipse,fill=ncentity!25,minimum height=2em,minimum width=8em,align=center},
  legend/.style={base, draw=none,rectangle},
  % coord node style is used for placing corners of connecting lines
  coord/.style={coordinate, on chain, on grid, node distance=6mm and 25mm},
  % nmark node style is used for coordinate debugging marks
  nmark/.style={draw, cyan, circle, font={\sffamily\bfseries}},
  % -------------------------------------------------
  % Connector line styles for different parts of the diagram
  author/.style={->, draw, lcauthor},
  adivsor/.style={->, draw, lcadvisor},
  use/.style={->, draw, lcuse},
  attr/.style={->, draw, lcattr},
  it/.style={font={\small\itshape}}
}
% -------------------------------------------------
% Legend
\node [legend] (l3) {Link:};
\node [legend, below=of l3, yshift=1.5em ] (l5)             {Link:};
\node [legend, below=of l5, yshift=1.5em ] (l7)             {Link:};
\node [legend, right= 30mm of l3] (l4)             {Generation};
\node [legend, right= 30mm of l5] (l6) {Attribution};
\node [legend, right= 30mm of l7] (l8) {Association};
% Start by placing the nodes
\node [entity, below=of l7, densely dotted, it] (e0) {\href{https://ideas.repec.org/p/cen/wpaper/10-11.html}{hdl:RePEc:cen:wpaper:10-11}};
% Use join to connect a node to the previous one 
\node [activity] (a1)     {act:research:12345};
\node [agent] (p2) {\href{https://ideas.repec.org/e/pab175.html}{repec:pab175}};
% We position the next block explicitly as the first block in the
% second column.  The chain 'comes along with us'. The distance
% between columns has already been defined, so we don't need to
% specify it.
\node [agent, , right=of a1] (p4) {\href{https://ideas.repec.org/e/pvi26.html}{repec:pvi26}};
% -------------------------------------------------
% Now we place the coordinate nodes for the connectors with angles, or
% with annotations. We also mark them for debugging.
\node [coord, left=40mm of e0] (c0)  {}; \cmark{0}   
\node [coord, left=40mm of p2] (c1)  {}; \cmark{1}   
\node [coord, right=60mm of e0] (c2)  {}; \cmark{2}   
%\node [coord, right=10mm of c2] (c3)  {}; \cmark{3}   
% -------------------------------------------------
% A couple of boxes have annotations
%\node [above=0mm of p4, it] {(Queue was empty)};
%\node [above=0mm of p8, it] {(Queue was not empty)};
% -------------------------------------------------
% All the other connections come out of tests and need annotating
% First, the straight north-south connections. In each case, we first
% draw a path with a (consistently positioned) annotation node, then
% we draw the arrow itself.
\path (e0.south) to node [left] {wasGeneratedBy} (a1);
  \draw [->,lcuse] (e0.south) -- (a1);
\path (a1.south) to node [left] {wasAssociatedWith} (p2);
  \draw [->,lcauthor] (a1.south) -- (p2);
\path (a1.east) to node [above] {wasAssociatedWith} (p4);
  \draw [->,lcauthor] (a1.east) -- (p4);
% Legend
\path (l3.east) to node {} (l4);
\draw [->,lcuse] (l3.east) -- (l4);
\path (l5.east) to node {} (l6);
\draw [->,lcattr] (l5.east) -- (l6);
\path (l7.east) to node {} (l8);
\draw [->,lcauthor] (l7.east) -- (l8);
% Attribution
\path (e0.west) to node {} (c0);
  \draw [-,lcattr] (e0.west) -- (c0);
\path (c0) to node  {wasAttributedTo} (c1);
  \draw [-,lcattr] (c0) -- (c1);
\path (c1) to node {} (p2.west);
  \draw [->,lcattr] (c1) -- (p2.west);
\path (e0.east) to node [above] {wasAttributedTo} (c2);
  \draw [-,lcattr] (e0.east) -- (c2);
\path (c2) to node {} (p4.north);
  \draw [->,lcattr] (c2) -- (p4.north);

%\path (t2.south) to node [near start, xshift=1em] {$y$} (t3); 
%  \draw [*->,lcauthor] (t2.south) -- (t3);
%\path (t3.south) to node [near start, xshift=1em] {$y$} (t4); 
%  \draw [*->,lcauthor] (t3.south) -- (t4);
%\path (t5.south) to node [near start, xshift=1em] {$y$} (t6); 
%  \draw [*->,lcadivsor] (t5.south) -- (t6);
%\path (t6.south) to node [near start, xshift=1em] {$y$} (t7); 
%  \draw [*->,lcadivsor] (t6.south) -- (t7); 
%% ------------------------------------------------- 
%% Now the straight east-west connections. To provide consistent
%% positioning of the test exit annotations, we have positioned
%% coordinates for the vertical part of the connectors. The annotation
%% text is positioned on a path to the coordinate, and then the whole
%% connector is drawn to its destination box.
%\path (t3.east) to node [near start, yshift=1em] {$n$} (c3); 
%  \draw [o->,lccong] (t3.east) -- (p8);
%\path (t4.east) to node [yshift=-1em] {$k \leq 0$} (c4r); 
%  \draw [o->,lcauthor] (t4.east) -- (p9);
%% -------------------------------------------------
%% Finally, the twisty connectors. Again, we place the annotation
%% first, then draw the connector
%\path (t1.east) to node [near start, yshift=1em] {$n$} (c1); 
%  \draw [o->,lcadivsor] (t1.east) -- (c1) |- (p4);
%\path (t2.east) -| node [very near start, yshift=1em] {$n$} (c1); 
%  \draw [o->,lcadivsor] (t2.east) -| (c1);
%\path (t4.west) to node [yshift=-1em] {$k>0$} (c4); 
%  \draw [*->,lcauthor] (t4.west) -- (c4) |- (p3);
%\path (t5.east) -| node [very near start, yshift=1em] {$n$} (c6); 
%  \draw [o->,lcadivsor] (t5.east) -| (c6); 
%\path (t6.east) to node [near start, yshift=1em] {$n$} (c6); 
%  \draw [o->,lcadivsor] (t6.east) -| (c7); 
%\path (t7.east) to node [yshift=-1em] {$k \leq 0$} (c7); 
%  \draw [o->,lcadivsor] (t7.east) -- (c7)  |- (p9);
%\path (t7.west) to node [yshift=-1em] {$k>0$} (c5); 
%  \draw [*->,lcadivsor] (t7.west) -- (c5) |- (p5);
% -------------------------------------------------
% -------------------------------------------------
\end{tikzpicture}

\end{figure}

\begin{figure}
\caption{Provenance back to data}\label{fig:workflow:complex}
% Flowcharting techniques for easy maintenance
% Author: Brent Longborough
% http://www.texample.net/media/tikz/examples/TEX/flexible-flow-chart.tex
% Start the picture
\begin{tikzpicture}[%
    >=triangle 60,              % Nice arrows; your taste may be different
    start chain=going below,    % General flow is top-to-bottom
    node distance=10mm and 60mm, % Global setup of box spacing
    every join/.style={author},   % Default linetype for connecting boxes
    ]
% ------------------------------------------------- 
% A few box styles 
% <on chain> *and* <on grid> reduce the need for manual relative
% positioning of nodes
\tikzset{
  base/.style={draw, on chain, on grid, align=center, minimum height=4ex},
  activity/.style={base, rectangle, text width=8em, minimum height=2em},
  agent/.style={base, diamond, aspect=2, text width=5em},
  entity/.style={base, ellipse,fill=ncentity!25,minimum height=2em,minimum width=8em,align=center},
  legend/.style={base, draw=none,rectangle},
  % coord node style is used for placing corners of connecting lines
  coord/.style={coordinate, on chain, on grid, node distance=6mm and 25mm},
  % nmark node style is used for coordinate debugging marks
  nmark/.style={draw, cyan, circle, font={\sffamily\bfseries}},
  % -------------------------------------------------
  % Connector line styles for different parts of the diagram
  author/.style={->, draw, lcauthor},
  adivsor/.style={->, draw, lcadvisor},
  use/.style={->, draw, lcuse},
  attr/.style={->, draw, lcattr},
  it/.style={font={\small\itshape}}
}
% -------------------------------------------------
% Legend
\node [legend] (l3) {Link:};
\node [legend, below=of l3, yshift=1.5em ] (l5)             {Link:};
\node [legend, below=of l5, yshift=1.5em ] (l7)             {Link:};
\node [legend, right= 30mm of l3] (l4)             {Generation/Usage};
\node [legend, right= 30mm of l5] (l6) {Attribution};
\node [legend, right= 30mm of l7] (l8) {Association};
% Start by placing the nodes
\node [entity, below=of l7, densely dotted, it] (e0) {\href{https://ideas.repec.org/p/cen/wpaper/10-11.html}{hdl:RePEc:cen:wpaper:10-11}};
% Use join to connect a node to the previous one 
\node [activity] (a1)     {act:writing:12345};
\node [entity, densely dotted, it] (e1) {file:home:spec555:path:to:file:dta};
\node [activity] (a2)     {act:research:12345};
\node [entity, densely dotted, it] (e2) {file:home:spec555:path:to:program:do};
% We position the next block explicitly as the first block in the
% second column.  The chain 'comes along with us'. The distance
% between columns has already been defined, so we don't need to
% specify it.
\node [agent, , right=of a1] (p4) {\href{https://ideas.repec.org/e/pvi26.html}{repec:pvi26}};
\node [agent] (p5) {\href{https://ideas.repec.org/e/pab175.html}{repec:pab175}};

% -------------------------------------------------
% Now we place the coordinate nodes for the connectors with angles, or
% with annotations. We also mark them for debugging.
\node [coord, left=40mm of e0] (c0)  {}; \cmark{0}   
\node [coord, left=40mm of p2] (c1)  {}; \cmark{1}   
\node [coord, right=60mm of e0] (c2)  {}; \cmark{2}   
\node [coord, right=20mm of c2] (c3)  {}; \cmark{3}   
\node [coord, right=20mm of p5] (c4)  {}; \cmark{4}   
%\node [coord, right=10mm of c2] (c3)  {}; \cmark{3}   
% -------------------------------------------------
% A couple of boxes have annotations
%\node [above=0mm of p4, it] {(Queue was empty)};
%\node [above=0mm of p8, it] {(Queue was not empty)};
% -------------------------------------------------
% All the other connections come out of tests and need annotating
% First, the straight north-south connections. In each case, we first
% draw a path with a (consistently positioned) annotation node, then
% we draw the arrow itself.
\path (e0.south) to node [left] {wasGeneratedBy} (a1);
  \draw [->,lcuse] (e0.south) -- (a1);
\path (e1.south) to node [left] {wasGeneratedBy} (a2);
  \draw [->,lcuse] (e1.south) -- (a2);

\path (a1.south) to node [left] {used} (e1);
  \draw [->,lcuse] (a1.south) -- (e1);
\path (a2.south) to node [left] {used} (e2);
  \draw [->,lcuse] (a2.south) -- (e2);

% Associations
\path (a1.east) to node [left] {} (p4);
  \draw [->,lcauthor] (a1.east) -- (p4);
\path (a1.east) to node [above] {wasAssociatedWith} (p5);
  \draw [->,lcauthor] (a1.east) -- (p5);
% Legend
\path (l3.east) to node {} (l4);
\draw [->,lcuse] (l3.east) -- (l4);
\path (l5.east) to node {} (l6);
\draw [->,lcattr] (l5.east) -- (l6);
\path (l7.east) to node {} (l8);
\draw [->,lcauthor] (l7.east) -- (l8);
% Attribution
%\path (e0.west) to node {} (c0);
%  \draw [-,lcattr] (e0.west) -- (c0);
%\path (c0) to node  {wasAttributedTo} (c1);
%  \draw [-,lcattr] (c0) -- (c1);
%\path (c1) to node {} (p2.west);
%  \draw [->,lcattr] (c1) -- (p2.west);
% Advisors
\path (e0.east) to node [above] {wasAttributedTo} (c2);
  \draw [-,lcadvisor] (e0.east) -- (c2);
\path (c2) to node {} (p4.north);
  \draw [->,lcadvisor] (c2) -- (p4.north);
\path (c2) to node {} (c3);
  \draw [-,lcadvisor] (c2) -- (c3);
\path (c3) to node {} (c4);
  \draw [-,lcadvisor] (c3) -- (c4);
\path (c4) to node {} (p5.east);
  \draw [->,lcadvisor] (c4) -- (p5.east);


%\path (t2.south) to node [near start, xshift=1em] {$y$} (t3); 
%  \draw [*->,lcauthor] (t2.south) -- (t3);
%\path (t3.south) to node [near start, xshift=1em] {$y$} (t4); 
%  \draw [*->,lcauthor] (t3.south) -- (t4);
%\path (t5.south) to node [near start, xshift=1em] {$y$} (t6); 
%  \draw [*->,lcadivsor] (t5.south) -- (t6);
%\path (t6.south) to node [near start, xshift=1em] {$y$} (t7); 
%  \draw [*->,lcadivsor] (t6.south) -- (t7); 
%% ------------------------------------------------- 
%% Now the straight east-west connections. To provide consistent
%% positioning of the test exit annotations, we have positioned
%% coordinates for the vertical part of the connectors. The annotation
%% text is positioned on a path to the coordinate, and then the whole
%% connector is drawn to its destination box.
%\path (t3.east) to node [near start, yshift=1em] {$n$} (c3); 
%  \draw [o->,lccong] (t3.east) -- (p8);
%\path (t4.east) to node [yshift=-1em] {$k \leq 0$} (c4r); 
%  \draw [o->,lcauthor] (t4.east) -- (p9);
%% -------------------------------------------------
%% Finally, the twisty connectors. Again, we place the annotation
%% first, then draw the connector
%\path (t1.east) to node [near start, yshift=1em] {$n$} (c1); 
%  \draw [o->,lcadivsor] (t1.east) -- (c1) |- (p4);
%\path (t2.east) -| node [very near start, yshift=1em] {$n$} (c1); 
%  \draw [o->,lcadivsor] (t2.east) -| (c1);
%\path (t4.west) to node [yshift=-1em] {$k>0$} (c4); 
%  \draw [*->,lcauthor] (t4.west) -- (c4) |- (p3);
%\path (t5.east) -| node [very near start, yshift=1em] {$n$} (c6); 
%  \draw [o->,lcadivsor] (t5.east) -| (c6); 
%\path (t6.east) to node [near start, yshift=1em] {$n$} (c6); 
%  \draw [o->,lcadivsor] (t6.east) -| (c7); 
%\path (t7.east) to node [yshift=-1em] {$k \leq 0$} (c7); 
%  \draw [o->,lcadivsor] (t7.east) -- (c7)  |- (p9);
%\path (t7.west) to node [yshift=-1em] {$k>0$} (c5); 
%  \draw [*->,lcadivsor] (t7.west) -- (c5) |- (p5);
% -------------------------------------------------
% -------------------------------------------------
\end{tikzpicture}

\end{figure}


\section{Combining the subgraphs}
\label{sec:combine}

Describe how we would link to other provenance chains.

\clearpage

\section{A simple example}
\label{sec:example}

To provide an example, we read in a portion of the U.S. Census Bureau's Census of Population 
and Housing Public Use Microdata Sample (Decennial PUMS). Specifically, we used the Alaska 
subset (file \texttt{DS2} or \texttt{DS0002}, depending on where it is referenced) \cite{pumsak}. 
The data file and a draft program  were obtained from the \ac{ICPSR}. After 
editing\footnote{The program as provided does not work, for two reasons: it is meant to be 
edited in its structure, and the commands included are sometimes erroneous.}, the program 
\texttt{01\_stata.do} (see Appendix~\ref{sec:stata1}) 
was run to obtain a table with the distribution of those identifying with a particular Alaska Native 
tribe in the population, and as a fraction of those with some Alaska Native mention. To do so, we 
tabulate both \texttt{RACE2}, which lists a variety of groupings, but also lists 4 Alaska Native 
categories (\texttt{31-34}: Alaskan Athabascan, Aleut, Eskimo, Tlingit-Haida alone), and 
\texttt{RACE1}, which allows for either "4 - Alaska Native alone" or 
"5 - American Indian and Alaska Native tribes specifies, and no other races".  We use 
\texttt{pweight} to construct the relevant table. The resulting table is Table~\ref{freq_specific_ak}.

\input{../samples/ICPSR-PUMS/freq_specific_ak.tex}

Expressed as PROV, the resulting provenance graph is represented by 
Figure~\ref{fig:author:example}. We choose to describe the entity ``file:person.dta'' (a dataset), 
rather than ``Table~\ref{freq_specific_ak}'', because the latter is a summary (a simple 
specialization) of the former.  [need to express this in the PROV]. This allows us to simplify the 
graph somewhat more, without loss of generality, to Figure~\ref{fig:author:example:simple}. 

\begin{figure}
\centering
\caption{Sample research activity with full provenance}\label{fig:author:example}
% Flowcharting techniques for easy maintenance
% Author: Brent Longborough
% http://www.texample.net/media/tikz/examples/TEX/flexible-flow-chart.tex
% Start the picture
\begin{tikzpicture}[%
    >=triangle 60,              % Nice arrows; your taste may be different
    start chain=going below,    % General flow is top-to-bottom
    node distance=10mm and 60mm, % Global setup of box spacing
    every join/.style={author},   % Default linetype for connecting boxes
    ]
% ------------------------------------------------- 
% A few box styles 
% <on chain> *and* <on grid> reduce the need for manual relative
% positioning of nodes
\tikzset{
  base/.style={draw, on chain, on grid, align=center, minimum height=4ex},
  activity/.style={base, rectangle, text width=8em, minimum height=2em},
  agent/.style={base, diamond, aspect=2, text width=5em},
  entity/.style={base, ellipse,fill=ncentity!25,minimum height=2em,minimum width=8em,align=center},
  legend/.style={base, draw=none,rectangle},
  % coord node style is used for placing corners of connecting lines
  coord/.style={coordinate, on chain, on grid, node distance=6mm and 25mm},
  % nmark node style is used for coordinate debugging marks
  nmark/.style={draw, cyan, circle, font={\sffamily\bfseries}},
  % -------------------------------------------------
  % Connector line styles for different parts of the diagram
  author/.style={->, draw, lcauthor},
  adivsor/.style={->, draw, lcadvisor},
  use/.style={->, draw, lcuse},
  attr/.style={->, draw, lcattr},
  it/.style={font={\small\itshape}}
}
% -------------------------------------------------
% Legend
%\node [legend] (l3) {Link:};
%\node [legend, below=of l3, yshift=1.5em ] (l5)             {Link:};
%\node [legend, below=of l5, yshift=1.5em ] (l7)             {Link:};
%\node [legend, right= 30mm of l3] (l4)             {Generation/Usage};
%\node [legend, right= 30mm of l5] (l6) {Attribution};
%\node [legend, right= 30mm of l7] (l8) {Association};
% Start by placing the nodes
\node [entity, densely dotted, it] (e0) {Table~\ref{fig:workflow:complex}};
% Use join to connect a node to the previous one 
\node [activity] (a1)     {act:writing:12345};
\node [entity, densely dotted, it] (e1) {file:person.dta};
\node [activity] (a2)     {act:research:12345};
\node [entity, densely dotted, it] (e2) 
{\href{https://github.com/ncrncornell/workflow/blob/f7acff773673289301c19a46789f25cb89d7b569/samples/ICPSR-PUMS/01_stata.do}%
{github:f7acff77}};
\node [entity, densely dotted, it, right=of a2] (e3) 
{\href{http://doi.org/10.3886/ICPSR13568.v1}{doi:ICPSR13568.v1}};
% We position the next block explicitly as the first block in the
% second column.  The chain 'comes along with us'. The distance
% between columns has already been defined, so we don't need to
% specify it.
\node [agent, , right=of a1] (p4) {\href{https://ideas.repec.org/e/pvi26.html}{repec:pvi26}};

% -------------------------------------------------
% Now we place the coordinate nodes for the connectors with angles, or
% with annotations. We also mark them for debugging.
%\node [coord, left=40mm of e0] (c0)  {}; \cmark{0}   
%\node [coord, left=40mm of p2] (c1)  {}; \cmark{1}   
\node [coord, right=60mm of e0] (c2)  {}; \cmark{2}   
%\node [coord, right=20mm of c2] (c3)  {}; \cmark{3}   
%\node [coord, right=20mm of p5] (c4)  {}; \cmark{4}   
%\node [coord, right=10mm of c2] (c3)  {}; \cmark{3}   
% -------------------------------------------------
% A couple of boxes have annotations
%\node [above=0mm of p4, it] {(Queue was empty)};
%\node [above=0mm of p8, it] {(Queue was not empty)};
% -------------------------------------------------
% All the other connections come out of tests and need annotating
% First, the straight north-south connections. In each case, we first
% draw a path with a (consistently positioned) annotation node, then
% we draw the arrow itself.
\path (e0.south) to node [left] {wasGeneratedBy} (a1);
  \draw [->,lcuse] (e0.south) -- (a1);
\path (e1.south) to node [left] {wasGeneratedBy} (a2);
  \draw [->,lcuse] (e1.south) -- (a2);
%\path (e1.east) to node [left] {wasDerivedFrom} (e3);
  %\draw [->,lcuse] (e1.east) -- (e3);

\path (a1.south) to node [left] {used} (e1);
  \draw [->,lcuse] (a1.south) -- (e1);
\path (a2.south) to node [left] {used} (e2);
  \draw [->,lcuse] (a2.south) -- (e2);
\path (a2.east) to node [below] {used} (e3);
  \draw [->,lcuse] (a2.east) -- (e3);

% Associations
\path (a1.east) to node [above] {wasAssociatedWith} (p4);
  \draw [->,lcauthor] (a1.east) -- (p4);
\path (a2.east) to node [left] {} (p4);
  \draw [->,lcauthor] (a2.east) -- (p4);
% Legend
%\path (l3.east) to node {} (l4);
%\draw [->,lcuse] (l3.east) -- (l4);
%\path (l5.east) to node {} (l6);
%\draw [->,lcattr] (l5.east) -- (l6);
%\path (l7.east) to node {} (l8);
%\draw [->,lcauthor] (l7.east) -- (l8);
% Attribution
%\path (e0.west) to node {} (c0);
%  \draw [-,lcattr] (e0.west) -- (c0);
%\path (c0) to node  {wasAttributedTo} (c1);
%  \draw [-,lcattr] (c0) -- (c1);
%\path (c1) to node {} (p2.west);
%  \draw [->,lcattr] (c1) -- (p2.west);
% Advisors
\path (e0.east) to node [above] {wasAttributedTo} (c2);
  \draw [-,lcadvisor] (e0.east) -- (c2);
\path (c2) to node {} (p4.north);
  \draw [->,lcadvisor] (c2) -- (p4.north);
%\path (c2) to node {} (c3);
%  \draw [-,lcadvisor] (c2) -- (c3);
%\path (c3) to node {} (c4);
%  \draw [-,lcadvisor] (c3) -- (c4);
%\path (c4) to node {} (p5.east);
%  \draw [->,lcadvisor] (c4) -- (p5.east);


%\path (t2.south) to node [near start, xshift=1em] {$y$} (t3); 
%  \draw [*->,lcauthor] (t2.south) -- (t3);
%\path (t3.south) to node [near start, xshift=1em] {$y$} (t4); 
%  \draw [*->,lcauthor] (t3.south) -- (t4);
%\path (t5.south) to node [near start, xshift=1em] {$y$} (t6); 
%  \draw [*->,lcadivsor] (t5.south) -- (t6);
%\path (t6.south) to node [near start, xshift=1em] {$y$} (t7); 
%  \draw [*->,lcadivsor] (t6.south) -- (t7); 
%% ------------------------------------------------- 
%% Now the straight east-west connections. To provide consistent
%% positioning of the test exit annotations, we have positioned
%% coordinates for the vertical part of the connectors. The annotation
%% text is positioned on a path to the coordinate, and then the whole
%% connector is drawn to its destination box.
%\path (t3.east) to node [near start, yshift=1em] {$n$} (c3); 
%  \draw [o->,lccong] (t3.east) -- (p8);
%\path (t4.east) to node [yshift=-1em] {$k \leq 0$} (c4r); 
%  \draw [o->,lcauthor] (t4.east) -- (p9);
%% -------------------------------------------------
%% Finally, the twisty connectors. Again, we place the annotation
%% first, then draw the connector
%\path (t1.east) to node [near start, yshift=1em] {$n$} (c1); 
%  \draw [o->,lcadivsor] (t1.east) -- (c1) |- (p4);
%\path (t2.east) -| node [very near start, yshift=1em] {$n$} (c1); 
%  \draw [o->,lcadivsor] (t2.east) -| (c1);
%\path (t4.west) to node [yshift=-1em] {$k>0$} (c4); 
%  \draw [*->,lcauthor] (t4.west) -- (c4) |- (p3);
%\path (t5.east) -| node [very near start, yshift=1em] {$n$} (c6); 
%  \draw [o->,lcadivsor] (t5.east) -| (c6); 
%\path (t6.east) to node [near start, yshift=1em] {$n$} (c6); 
%  \draw [o->,lcadivsor] (t6.east) -| (c7); 
%\path (t7.east) to node [yshift=-1em] {$k \leq 0$} (c7); 
%  \draw [o->,lcadivsor] (t7.east) -- (c7)  |- (p9);
%\path (t7.west) to node [yshift=-1em] {$k>0$} (c5); 
%  \draw [*->,lcadivsor] (t7.west) -- (c5) |- (p5);
% -------------------------------------------------
% -------------------------------------------------
\end{tikzpicture}

\end{figure}

\begin{figure}
\centering
\caption{Sample research activity with simplified provenance}\label{fig:author:example:simple}
% Flowcharting techniques for easy maintenance
% Author: Brent Longborough
% http://www.texample.net/media/tikz/examples/TEX/flexible-flow-chart.tex
% Start the picture
\begin{tikzpicture}[%
    >=triangle 60,              % Nice arrows; your taste may be different
    start chain=going below,    % General flow is top-to-bottom
    node distance=10mm and 60mm, % Global setup of box spacing
    every join/.style={author},   % Default linetype for connecting boxes
    ]
% ------------------------------------------------- 
% A few box styles 
% <on chain> *and* <on grid> reduce the need for manual relative
% positioning of nodes
\tikzset{
  base/.style={draw, on chain, on grid, align=center, minimum height=4ex},
  activity/.style={base, rectangle, text width=8em, minimum height=2em},
  agent/.style={base, diamond, aspect=2, text width=5em},
  entity/.style={base, ellipse,fill=ncentity!25,minimum height=2em,minimum width=8em,align=center},
  legend/.style={base, draw=none,rectangle},
  % coord node style is used for placing corners of connecting lines
  coord/.style={coordinate, on chain, on grid, node distance=6mm and 25mm},
  % nmark node style is used for coordinate debugging marks
  nmark/.style={draw, cyan, circle, font={\sffamily\bfseries}},
  % -------------------------------------------------
  % Connector line styles for different parts of the diagram
  author/.style={->, draw, lcauthor},
  adivsor/.style={->, draw, lcadvisor},
  use/.style={->, draw, lcuse},
  attr/.style={->, draw, lcattr},
  it/.style={font={\small\itshape}}
}
% -------------------------------------------------
% Legend
%\node [legend] (l3) {Link:};
%\node [legend, below=of l3, yshift=1.5em ] (l5)             {Link:};
%\node [legend, below=of l5, yshift=1.5em ] (l7)             {Link:};
%\node [legend, right= 30mm of l3] (l4)             {Generation/Usage};
%\node [legend, right= 30mm of l5] (l6) {Attribution};
%\node [legend, right= 30mm of l7] (l8) {Association};
% Start by placing the nodes
%\node [entity, densely dotted, it] (e0) {Table~\ref{fig:workflow:complex}};
% Use join to connect a node to the previous one 
%\node [activity] (a1)     {act:writing:12345};
\node [entity, densely dotted, it] (e1) {file:person.dta};
\node [activity] (a2)     {act:research:12345};
\node [entity, densely dotted, it] (e2) 
{\href{https://github.com/ncrncornell/workflow/blob/f7acff773673289301c19a46789f25cb89d7b569/samples/ICPSR-PUMS/01_stata.do}%
{github:f7acff77}};
\node [entity, densely dotted, it, right=of e2] (e3) 
{\href{http://doi.org/10.3886/ICPSR13568.v1}{doi:ICPSR13568.v1}};
% We position the next block explicitly as the first block in the
% second column.  The chain 'comes along with us'. The distance
% between columns has already been defined, so we don't need to
% specify it.
\node [agent, , right=of a1] (p4) {\href{https://ideas.repec.org/e/pvi26.html}{repec:pvi26}};

% -------------------------------------------------
% Now we place the coordinate nodes for the connectors with angles, or
% with annotations. We also mark them for debugging.
%\node [coord, left=40mm of e0] (c0)  {}; \cmark{0}   
%\node [coord, left=40mm of p2] (c1)  {}; \cmark{1}   
%\node [coord, right=60mm of e0] (c2)  {}; \cmark{2}   
%\node [coord, right=20mm of c2] (c3)  {}; \cmark{3}   
%\node [coord, right=20mm of p5] (c4)  {}; \cmark{4}   
%\node [coord, right=10mm of c2] (c3)  {}; \cmark{3}   
% -------------------------------------------------
% A couple of boxes have annotations
%\node [above=0mm of p4, it] {(Queue was empty)};
%\node [above=0mm of p8, it] {(Queue was not empty)};
% -------------------------------------------------
% All the other connections come out of tests and need annotating
% First, the straight north-south connections. In each case, we first
% draw a path with a (consistently positioned) annotation node, then
% we draw the arrow itself.
%\path (e0.south) to node [left] {wasGeneratedBy} (a1);
 % \draw [->,lcuse] (e0.south) -- (a1);
\path (e1.south) to node [left] {wasGeneratedBy} (a2);
  \draw [->,lcuse] (e1.south) -- (a2);
%\path (e1.east) to node [left] {wasDerivedFrom} (e3);
  %\draw [->,lcuse] (e1.east) -- (e3);

%\path (a1.south) to node [left] {used} (e1);
 % \draw [->,lcuse] (a1.south) -- (e1);
\path (a2.south) to node [left] {used} (e2);
  \draw [->,lcuse] (a2.south) -- (e2);
\path (a2.east) to node [below] {used} (e3);
  \draw [->,lcuse] (a2.east) -- (e3);

% Associations
%\path (a1.east) to node [above] {wasAssociatedWith} (p4);
% \draw [->,lcauthor] (a1.east) -- (p4);
\path (a2.east) to node [above] {wasAssociatedWith} (p4);
  \draw [->,lcauthor] (a2.east) -- (p4);
% Legend
%\path (l3.east) to node {} (l4);
%\draw [->,lcuse] (l3.east) -- (l4);
%\path (l5.east) to node {} (l6);
%\draw [->,lcattr] (l5.east) -- (l6);
%\path (l7.east) to node {} (l8);
%\draw [->,lcauthor] (l7.east) -- (l8);
% Attribution
%\path (e0.west) to node {} (c0);
%  \draw [-,lcattr] (e0.west) -- (c0);
%\path (c0) to node  {wasAttributedTo} (c1);
%  \draw [-,lcattr] (c0) -- (c1);
%\path (c1) to node {} (p2.west);
%  \draw [->,lcattr] (c1) -- (p2.west);
% Advisors
%\path (e0.east) to node [above] {wasAttributedTo} (c2);
 % \draw [-,lcadvisor] (e0.east) -- (c2);
%\path (c2) to node {} (p4.north);
 % \draw [->,lcadvisor] (c2) -- (p4.north);
%\path (c2) to node {} (c3);
%  \draw [-,lcadvisor] (c2) -- (c3);
%\path (c3) to node {} (c4);
%  \draw [-,lcadvisor] (c3) -- (c4);
%\path (c4) to node {} (p5.east);
%  \draw [->,lcadvisor] (c4) -- (p5.east);


%\path (t2.south) to node [near start, xshift=1em] {$y$} (t3); 
%  \draw [*->,lcauthor] (t2.south) -- (t3);
%\path (t3.south) to node [near start, xshift=1em] {$y$} (t4); 
%  \draw [*->,lcauthor] (t3.south) -- (t4);
%\path (t5.south) to node [near start, xshift=1em] {$y$} (t6); 
%  \draw [*->,lcadivsor] (t5.south) -- (t6);
%\path (t6.south) to node [near start, xshift=1em] {$y$} (t7); 
%  \draw [*->,lcadivsor] (t6.south) -- (t7); 
%% ------------------------------------------------- 
%% Now the straight east-west connections. To provide consistent
%% positioning of the test exit annotations, we have positioned
%% coordinates for the vertical part of the connectors. The annotation
%% text is positioned on a path to the coordinate, and then the whole
%% connector is drawn to its destination box.
%\path (t3.east) to node [near start, yshift=1em] {$n$} (c3); 
%  \draw [o->,lccong] (t3.east) -- (p8);
%\path (t4.east) to node [yshift=-1em] {$k \leq 0$} (c4r); 
%  \draw [o->,lcauthor] (t4.east) -- (p9);
%% -------------------------------------------------
%% Finally, the twisty connectors. Again, we place the annotation
%% first, then draw the connector
%\path (t1.east) to node [near start, yshift=1em] {$n$} (c1); 
%  \draw [o->,lcadivsor] (t1.east) -- (c1) |- (p4);
%\path (t2.east) -| node [very near start, yshift=1em] {$n$} (c1); 
%  \draw [o->,lcadivsor] (t2.east) -| (c1);
%\path (t4.west) to node [yshift=-1em] {$k>0$} (c4); 
%  \draw [*->,lcauthor] (t4.west) -- (c4) |- (p3);
%\path (t5.east) -| node [very near start, yshift=1em] {$n$} (c6); 
%  \draw [o->,lcadivsor] (t5.east) -| (c6); 
%\path (t6.east) to node [near start, yshift=1em] {$n$} (c6); 
%  \draw [o->,lcadivsor] (t6.east) -| (c7); 
%\path (t7.east) to node [yshift=-1em] {$k \leq 0$} (c7); 
%  \draw [o->,lcadivsor] (t7.east) -- (c7)  |- (p9);
%\path (t7.west) to node [yshift=-1em] {$k>0$} (c5); 
%  \draw [*->,lcadivsor] (t7.west) -- (c5) |- (p5);
% -------------------------------------------------
% -------------------------------------------------
\end{tikzpicture}

\end{figure}

The full PROV associated with this graph is depicted in Appendix~\ref{sec:example:prov}.

\section{An graphical interface}
\label{sec:interface}
%Here plug in the graphical interface and tool. See example at \url{http://dev.ncrn.cornell.edu/ced2ar-web/prov2/}.

The graphical workflow tool elicits all the information needed in Figure~\ref{fig:workflow:complex} from the user in a flexible manner, starting with an initial blank template (Figure~\ref{fig:ui:initial}). The user can then edit the information on entities (Figure~\ref{fig:ui:edit}). Information on outside datasets can be prompted using selectors, for known or user-provided repositories, dynamically loading in information using supported protocols (OAI-PMH, DDI-Disco) (Figure~\ref{fig:ui:data-selector:doi}). Information on internal entities (programs, unpublished datasets) are elicited from the user, using a file browser, Figure~\ref{fig:ui:data-selector:local} (caching previous runs of the application will also be explored). \marginpar{I don't know how to get to that selector from an existing node}. Once selected, the user is encouraged to provide a full documentation for any datasets through the integrated DDI editor. Additional entities can be added to make more complex workflows (Figure)The PROV serves both as a guide to the workflow tool, as well as a standards-compliant documentation of the provenance of the research article.

\begin{figure}
\caption{Initial user interface and entity editing}\label{fig:ui:initial}
\centering
\includegraphics[width=0.45\textwidth]{figure-ced2ar-ui-initial}
\label{fig:ui:edit}
\centering
\includegraphics[width=0.45\textwidth]{figure-ced2ar-ui-edit}
\end{figure}



\begin{figure}
\caption{Selector for external and local entities}\label{fig:ui:data-selector:doi}
\label{fig:ui:data-selector:local}
\includegraphics[width=0.45\textwidth]{figure-ced2ar-ui-local-selector}
\includegraphics[width=0.45\textwidth]{figure-ced2ar-ui-remote-selector}

\end{figure}

\begin{figure}
\caption{Adding entities}\label{fig:ui:add-entity}
\centering
\includegraphics[width=0.5\textwidth]{figure-ced2ar-ui-add-entity}
\end{figure}

\marginpar{OK, how do we go from here to editing the DDI?}

\clearpage

\section{Future work}
\label{sec:future_work}
Already present in the \repec~ network are citation links (linking papers among themselves). This work links in with CED$^2$AR work (citations) and other efforts for linking papers and articles to the data used for (empirical) papers. Establishing such links can be represented by a tripartite graph:

\printbibliography

\appendix

\section{Stata program for readin of PUMS as used}
\label{sec:stata1}
The following program was run to obtain Table~\ref{freq_specific_ak}. The program can be run 
in batch mode (\texttt{stata -b do 01\_stata.do}), and will generate the table included above 
dynamically. It was derived from the (non-functioning) program and layout provided at 
\url{http://doi.org/10.3886/ICPSR13568.v1} (and archived at 
\url{https://github.com/ncrncornell/workflow/tree/master/samples/ICPSR-PUMS/ICPSR_13568}).

\lstinputlisting[basicstyle=\footnotesize,language=SAS,frame=single,keepspaces=false,breaklines=false,showspaces=false]{../samples/ICPSR-PUMS/01_stata.do}

%\section{Stata program for readin of PUMS provided by ICPSR}
%\ac{ICPSR} provides a Stata program, but it cannot be used as-is, as it needs to be edited, and 
%Stata commands need to be corrected. 
%\lstinputlisting[language=XML,breaklines=true]{../samples/ICPSR-PUMS/ICPSR_13568/13568-Setup.do}

\section{PROV for simple example of Section~\ref{sec:example}}
\label{sec:example:prov}
The following omits the first document declaration for clarity. The complete file 
\textattachfile{../samples/ICPSR-PUMS/pumsak.dta.ddi25-ncrn.xml}{is attached}.
\lstinputlisting[language=XML,breaklines=true,firstline=52,lastline=97]{../samples/ICPSR-PUMS/pumsak.dta.ddi25-ncrn.xml}

\section{Additional processing}
For the purpose of this paper, 
 we used OpenDataForge Sledgehammer to convert the Stata file produced by the program in 
Appendix~\ref{sec:stata1} to DDI 2.5. In order to use the freeware version of Sledgehammer, we subset the file to 4999 
observations. 
\lstinputlisting[frame=single,showspaces=false,language=sh]{../samples/ICPSR-PUMS/sledgehammer.out}


\section*{Acronyms used}
%TCIDATA{Version=5.00.0.2570}
%TCIDATA{LaTeXparent=0,0,sw-edit.tex}

% $Id: acronyms.tex 1258 2014-12-10 04:54:56Z lv39 $
% $URL: https://forge.cornell.edu/svn/repos/ncrn-cornell/branches/workflow/text/acronyms.tex $
%
% Define acronyms to be used in the text here. See
% http://www.mackichan.com/index.html?techtalk/456.htm~mainFrame for usage in
% Scientific workplace context

\begin{acronym}
\acro{ACS}{American Community Survey}
\acro{BDS}{Business Dynamics Statistics}
\acro{BED}{Business Employment Dynamics}
\acro{BLS}{Bureau of Labor Statistics}
\acro{CAC}{Cornell Center for Advanced Computing}
\acro{CES}{Center for Economic Studies}
\acro{CIT}{Cornell Information Technologies}
\acro{CISER}{Cornell Institute for Social and Economic Research}
\acro{CRADC}{Cornell Restricted Access Data Center}
\acro{CTC}{Cornell Theory Center}
\acro{CER}{Covered Earnings Records}
\acro{CPS}{Current Population Survey}
\acro{DCC}{Data Confidentiality Committee}
\acro{DDI}{Data Documentation Initiative\acroextra{, see \href{http://www.ddialliance.org/}{http://www.ddialliance.org/}}}
\acro{DOI}{Digital Object Identifier}
\acro{DER}{Detailed Earnings Record}
\acro{ERR}{Excess Reallocation Rate}
\acrodef{err}{excess reallocation rate}
\acro{HRS}{Health and Retirement Study}
\acro{ICPSR}{Inter-university Consortium for Political and Social Research}
\acro{IDSC}{International Data Service Center}
\acro{IPUMS}{Integrated Public Use Microdata Series}
\acro{IRB}{Institutional Review Board}
\acro{IRS}{Internal Revenue Service}
\acro{ISBN}{International Standard Book Number}
\acro{ISSN}{International Standard Serial Number}
\acro{ISR}{Institute for Social Research}
\acro{IZA}{Institute for the Study of Labor}
\acro{JOLTS}{Job Openings and Labor Turnover Survey}
\acro{JCR}{Job Creation Rate}
\acro{JDR}{Job Destruction Rate}
\acro{JRR}{Job Reallocation Rate}
\acrodef{jcr}{job creation rate}
\acrodef{jdr}{job destruction rate}
\acrodef{jrr}{job reallocation rate}
\acro{LBD}{Longitudinal Business Database}
\acro{LDB}{\ac{BLS}'s Longitudinal Business Database}
\acro{LED}{Local Employment Dynamics}
\acro{LEHD}{Longitudinal Employer-Household Dynamics}
\acro{MBR}{Master Beneficiary Record}
\acro{MEF}{Master Earnings File}
\acro{MER}{Master Earnings Record}
\acro{MMS}{Methodology, Measurement, and Statistics}
\acro{NIA}{National Institute on Aging}
\acro{NORC}{NORC}
\acro{NSF}{National Science Foundation}
\acro{OTM}{OnTheMap}
\acro{QWI}{Quarterly Workforce Indicators}
\acro{QCEW}{Quarterly Census of Employment and Wages}
\acro{RDA}{Restricted Data Application}
\acro{RDC}{Research Data Center}
\acro{SDMX}{Statistical Data and Metadata eXchange\acroextra{, see \href{http://sdmx.org/}{http://sdmx.org}}}
\acro{SEPB}{Summary of Earnings and Projected Benefits} % confidential SSA                                % file
\acro{SIPP}{Survey of Income and Program Participation}
\acro{SLID}{Survey of Labour and Income Dynamics}
\acro{SRMI}{Sequential Regression Multivariate Imputation}
\acro{SSA}{Social Security Administration}
\acro{SSI}{Supplemental Security Income}
\acro{SSN}{Social Security Number}
\acro{SSR}{Supplemental Security Record}
\acro{UI}{Unemployment Insurance}
\acro{WRR}{Worker Reallocation Rate}
\acrodef{wrr}{worker reallocation rate}
\acro{WTS}{Windows Terminal Services}
\acro{XML}{Extensible Markup Language}
\acro{CODA}{Children of Depression}
\acro{WB}{War Babies}
\acro{AHEAD}{Study of Assets and Health Dynamics Amongst the Oldest Old}
% Usage in the later text:
%  \ac{acronym}         Expand and identify the acronym the first time; use
%                       only the acronym thereafter
%  \acf{acronym}        Use the full name of the acronym.
%  \acs{acronym}        Use the acronym, even before the first corresponding
%                       \ac command
%  \acl{acronym}        Expand the acronym without using the acronym itself.
\end{acronym}

%%% Local Variables:
%%% mode: latex
%%% TeX-master: "proposal"
%%% End:

\end{document}
