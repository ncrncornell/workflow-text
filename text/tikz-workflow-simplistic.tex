% Flowcharting techniques for easy maintenance
% Author: Brent Longborough
% http://www.texample.net/media/tikz/examples/TEX/flexible-flow-chart.tex
% Start the picture
\begin{tikzpicture}[%
    >=triangle 60,              % Nice arrows; your taste may be different
    start chain=going below,    % General flow is top-to-bottom
    node distance=10mm and 60mm, % Global setup of box spacing
    every join/.style={author},   % Default linetype for connecting boxes
    ]
% ------------------------------------------------- 
% A few box styles 
% <on chain> *and* <on grid> reduce the need for manual relative
% positioning of nodes
\tikzset{
  base/.style={draw, on chain, on grid, align=center, minimum height=4ex},
  activity/.style={base, rectangle, text width=8em, minimum height=2em},
  agent/.style={base, diamond, aspect=2, text width=5em},
  entity/.style={base, ellipse,fill=ncentity!25,minimum height=2em,minimum width=8em,align=center},
  legend/.style={base, draw=none,rectangle},
  % coord node style is used for placing corners of connecting lines
  coord/.style={coordinate, on chain, on grid, node distance=6mm and 25mm},
  % nmark node style is used for coordinate debugging marks
  nmark/.style={draw, cyan, circle, font={\sffamily\bfseries}},
  % -------------------------------------------------
  % Connector line styles for different parts of the diagram
  author/.style={->, draw, lcauthor},
  adivsor/.style={->, draw, lcadvisor},
  use/.style={->, draw, lcuse},
  attr/.style={->, draw, lcattr},
  it/.style={font={\small\itshape}}
}
% -------------------------------------------------
% Legend
% Start by placing the nodes
\node [entity, densely dotted, it] (e0) {Paper};
% Use join to connect a node to the previous one 
\node [entity, densely dotted, it] (e1) {Table 1};
\node [entity, densely dotted, it] (e2) {program.do};
\node [entity, densely dotted, it] (e3) {CPS};
% We position the next block explicitly as the first block in the
% second column.  The chain 'comes along with us'. The distance
% between columns has already been defined, so we don't need to
% specify it.
\node [agent, , right=of e1] (p0) {Researcher};


% -------------------------------------------------
% Now we place the coordinate nodes for the connectors with angles, or
% with annotations. We also mark them for debugging.
%\node [coord, left=40mm of e0] (c0)  {}; \cmark{0}   
%\node [coord, left=40mm of p2] (c1)  {}; \cmark{1}   
%\node [coord, right=60mm of e0] (c2)  {}; \cmark{2}   
%\node [coord, right=20mm of c2] (c3)  {}; \cmark{3}   
%\node [coord, right=20mm of p5] (c4)  {}; \cmark{4}   
%\node [coord, right=10mm of c2] (c3)  {}; \cmark{3}   
% -------------------------------------------------
% A couple of boxes have annotations
%\node [above=0mm of p4, it] {(Queue was empty)};
%\node [above=0mm of p8, it] {(Queue was not empty)};
% -------------------------------------------------
% All the other connections come out of tests and need annotating
% First, the straight north-south connections. In each case, we first
% draw a path with a (consistently positioned) annotation node, then
% we draw the arrow itself.
\path (e0.south) to node [left] {contains} (e1);
  \draw [->,lcuse] (e0.south) -- (e1);
\path (e1.south) to node [left] {produced by} (e2);
  \draw [->,lcuse] (e1.south) -- (e2);
\path (e2.south) to node [left] {used} (e3);
  \draw [->,lcuse] (e2.south) -- (e3);
\end{tikzpicture}
